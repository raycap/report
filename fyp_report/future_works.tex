\chapter{Conclusion and Future Works}
\section{Conclusion}
This project studies about contact force estimation using motor currents. This project specifically study about the Denso VS-060 arm. Previous works have been reviewed about many advance approaches that has been researched to estimate the contact forces without force sensor. Due to some limitations, not all of the methods can be applied for this project. 

A basic dynamic system was used as a fundamental model to estimate the contact force. A friction model was needed in the model and two basic friction were choosen: viscous and Coulomb model (static friction) and Dahl model (dynamic friction). To be able to estimate contact force without force / torque sensor, identification of unknown parameters of the system is needed. In general, two parameters were to be identified, they are: friction and gain parameter from denso motor currents / torques to real value of joint torque from motor. 

Before continue with experiments, problem regarding motor currents were present and hence it had to be solved first. The problem was that the motor currents would always give positive value, thus a way to determine the sign of the currents was needed. A solution to this was to use motor torques as it also had the sign value along with the number. Hence, there is no need to use motor currents anymore as motor torques are now available. 

Thereafter, experiments were performed. The model identification then had been done and verified in the early stage. However, as the experiment was not structurized, it was then retried using more structured setup which is two-stage experiment. Unfortunately, there was a problem regarding the OpenRave that obstruct the process of identification. Hence, the identification could not be completed and validation had not been checked for the latest experiments.

\section{Future Works}
The future works for this project may include: troubleshoot of the current problem of the OpenRave or perform identification of robot inertial parameters, finish the development and validation of the model using two-stage experiments, test the developed model to estimate the contact force for manipulator arm in real time experiment, and start controlling the robot to do force control using the developed system. 



