\chapter{Conclusion and Future Works}
\section{Conclusion}
This project studied about contact force control using only motor currents / torques. Hence force estimation was the most crucial part in this project. This project specifically studied about the Denso VS-060 arm. Previous works had been reviewed about many advance approaches that had been researched to estimate the contact forces without force sensor. Due to some limitations, not all of the methods could be applied for this project. 

A basic dynamic system was used as a fundamental model to estimate the contact force. A friction model was needed in the model and two basic friction were chosen: viscous and Coulomb model (static friction) and Dahl model (dynamic friction). To be able to estimate contact force without force / torque sensor, identification of unknown parameters of the system was needed. In general, two parameters were to be identified, they are: friction and gain parameter from denso motor currents / torques to real value of joint torque from motor. 

Before continue with experiments, problem regarding motor currents were present and hence it had to be solved first. The problem was that the motor currents would always give positive value, thus a way to determine the sign of the currents was needed. A solution to this was to use motor torques as it also had the sign value along with the number. Hence, there is no need to use motor currents anymore as motor torques are now available. 

Thereafter, experiments were performed. The model identification then had been done and verified in the early stage. However, as the experiment was not structurized and the results were difficult to be identified, it was then retried using more structured and easier setup which was two-stage experiment. All necessary parameters were then recalculated.

Once all of the parameters had been identified, the algorithm to estimate the contact force was created. The model would estimate the force based on only motor torques and joint velocities. The computation also assumed quasistatic model for simplicity.

The next step was to validate the developed model estimation with the F/T sensor using new data. Based on data, it was believed that the results were more reasonable once contact forces had been introduced. 

Finally, a force control based on the algorithm was constructed, Then a small assembly task which is pin insertion was performed using this force controller to control the manipulator. After several tuning of PD gain the manipulator was able to successfuly insert the pin into the hole.

\section{Future Works}
The future works for this project may include: troubleshoot of the current problem of the OpenRave or perform identification of robot inertial parameters and improve the model to have a better estimation of contact force. 


