\chapter{Introduction}
\section{General Introduction}

Nowadays, robotics play a very crucial role in industrial area by greatly increasing the industrial productivity. It helps factory workers on doing many monotonous and tedious taks such as pick and place and welding operation. However, this achievement is done because of a highly structured environment such as heavy industry (for example car assembly) where every parameters are known and fixed. In contrast, robotics performance in light industry are still poor. For instances, assembly of small and fragile parts in electronics, food, and other industries. This is because robotics are still bad in dealing with dynamics and unstructured environment where uncertainities are common. 

\begin{figure}[h]
  \begin{subfigure}[t]{0.5\textwidth}
    \includegraphics[width = \textwidth ]{placeholder} 
    \caption{Robotics in heavy industries}
  \end{subfigure}
  \begin{subfigure}[t]{0.5\textwidth}{
    \includegraphics[width = \textwidth ]{placeholder}
    \caption{Manipulators for assembly task}}
  \end{subfigure}
  \caption{Robotics in some applications}
\end{figure}


In light of this, researches and developments in this topic are still intense until now. Many works have attempted to create a framework for fine assembly procedures. Recently, a paper by \cite{aaaa} has introduced the complete framework for fine assembly task. However, there are still lots of improvement that can be done.


For robotic assembly task, the robotic arm must be able to cooperate with a lot of uncertainites in the dynamic environment. One example, the manipulator has to be able to know when the contact with an object is happening and then maintaining the stable contact throughout the task. This ability requires knowledge of contact force for the robot. Thus, estimation of contact forces is very important since it will help the robot to determine and control the contact with objects in dynamics environment. While this can be done using accurate force/torque sensor, the sensor is normally expensive and requires mechanical integration with the robotic arm \cite{Hao15}. Hence in some arms it might not be possible to attach this sensor.  
 
In regards to this, many researches have been done in order to estimate the contact force. The main idea to estimate the contact force is to directly apply dynamic equations of a robot, knowing the value of joint torques. However different approaches have also been explored in the past few decades. Early approaches use observers for force estimation like in \cite{Ohi91}. Another approach in \cite{Stolt12} involves detune the low-level joint position control loop to estimate contact force. Furthermore, recent approaches by using Bayesian approach and generalized momentum with Kalman filter are studied in \cite{Hao14} and \cite{Hao15} respectively. Additionally, studies of comparison between two different approaches are done in \cite{Beyl11}. The study compares the result from filtered dynamic equation of external force with generalized momentum method.

\section{Objective}


This project aims to estimate the contact force of an assembly robot based on the arm motor currents/torques. Understanding of the mathematical model of the robot’s dynamic, friction, and control theory are considered as important knowledge to work with this project.

The project will be focusing on a certain Denso arm. Thus, the developed systems will be built specifically for this arm. Additionally, some problems that are discussed in this project will be only addressed for this Denso arm and might not be available for other arms.


\section{Scope}


The scope of this project is divided into four parts. First, the project will start from understanding of the general models of robot dynamic and friction. Thus, literature reviews and readings are included in this step. The next step is to perform some experiments to get all necessary data to develop the model. This includes setup preparations, running the experiments, and collections of the data. The third step will be processing all the results and develop the system to estimate the contact force. which after that, validation of the built model to the real data will be the final step.
